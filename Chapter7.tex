\chapter{Conclusions and Perspectives}
\label{chap:conclusions_perspectives}
%\section{Verifying the biological accuracy of the model}
%\label{sec:model_conclusions}
%SWALES:
% 1) Why interesting / important?
% 2) What's been done so far?
% 3) What's the hole
% 4) What you did
% 5) What you found
% 6) What it means

Existing models of the \ac{STG} include at most four or five neurons, with neurons of the same type, such as the \acp{PY} and \acp{PD}, modelled as a single neuron. With such a configuration it is not possible to model the differential modulation which occurs when \ac{STG} neurons are exposed to \ac{DA}.

This thesis addresses the need for a computational model to better understand the effect of \ac{DA} on individual neurons as part of a larger network. For this research our interest was focused on the \acp{PY} which comprise the pyloric \ac{CPG} in the \ac{STG} of \species{Cancer pagurus}. We have a certain expectation of the network output which is based on previous research. It has been shown that the \acp{PY} which are synchronised under normal conditions become desynchronised when exposed to \ac{DA} \note{\cite{Johnson1993}}.

This project was started with the gathering of experimental data to inform the computational model. Research into the crustacean \ac{STNS} has mainly been done using traditional electrophysiological methods such as intra-cellular recordings made with glass micro-electrodes that are inserted into the soma of the cell, and extra-cellular recordings that are made using suction or wire electrodes. Previous research at the lab of Prof. Peter Andras has shown that \ac{VSD} can be used successfully on the neurons of the \ac{STG}. The use of \ac{VSD} opens the possibility of recording all the neurons of the \ac{STG} at the same time. Such recordings allow us quantify the contribution of individual neurons to the \ac{CPG} rhythms and, in turn, model the network at neuron level.

To find an improved means of recording from the whole \ac{STG}, a great deal of time and effort was spent making \ac{VSD} recordings and investigating alternative means of recording. Analysis of the pyloric \ac{CPG} activity starts with the identification of the beginning of the pyloric rhythm. The rhythm is usually easily identifiable on extracellular recordings over the \ac{lvn} or \ac{dvn}. Identifying the rhythm was done by finding the first \ac{LP} spike\footnote{Physiologically the start of the pyloric rhythm would be the \ac{PD} activity as it is a pacemaker \cite{Harris-Warrick1992}. However, visually it was easier to use the \ac{LP} activity for this purpose. For the sake of analysis and timing it did not matter which neurons were used to identify the beginning of a phase as long as it is used consistently}. The \ac{LP} spikes are usually easily identifiable as the first long spike after a short pause. \ac{DA}, however, affects the amplitudes of the spikes and it often becomes impossible to identify the rhythm in this way. This significantly complicates identifying the phases of the rhythm. A possible solution to this relies on intracellular recordings, made either by using electrophysiology or \ac{VSD}. If at least one pyloric neuron can be positively identified, the beginning of its depolarisation period can be taken as the beginning of the phase. It has been found in \ac{VSD} imaging that the \ac{PD} signal usually dominates the neuropil. If this signal is found to form a recognisable wave it can be used to identify the beginning of the rhythm.

More detailed analysis of extracellular recordings made simultaneously over the \ac{lvn} and \ac{pyn} can also be useful, at least until other methods for recording and the data analysis for these methods become more mature. Such recordings would allow the identification of the individual \acp{PY} in the rhythm. Dissecting the \acp{pyn} requires some skill as the nerves tend to be very fine and can be difficult to dissect from the surrounding tissue in which it is embedded.

To verify the validity of a model we need to show that the model output is reflective of that of the biological system. We thus need to find means of quantifying the output of the biological system. We have described the detrending and triggered averaging methods used for analysing extracellular recordings and we have also proposed new methods for the analysis of \ac{VSD} recordings. The new method works off-line and involves the identification of salient features which are used to profile a waveform specific to a neuron. The features on  profiled waveforms are then used to quantify the extent of de-synchronisation of the neurons of interest, which in our case are the \acp{PY}.

The new analysis methods were successfully applied to recordings made from both bath-applied neurons and neurons that were individually injected with \ac{VSD}. While the methods have been shown to work well off-line, i.e. applied after the recordings have been made, it would be very useful if the methods can be extended to be used on-line. An on-line application would allow for the immediate identification of neurons that, in turn, would allow for the stimulation of individual neurons if the \ac{VSD} is used in conjunction with \ac{MEA} or traditional electrophysiological techniques.

Using established methodology and newly developed methods for analysis of the data as detailed in chapter \ref{chap:methodsAndMaterials} and \ref{chap:analysis} we were able to construct a detailed computational model of the \acp{PY} in the \ac{STG} that would reflect the expected desynchronisation. Hodkin-Huxley's conductance-based, mathematical model was used as the basis for developing the model for this research. Each neuron is represented with two compartments, the soma and the axon. All axons include three channels, $Na$, $K$ and $L$ while the soma had, depending on the neuron type, up to nine channels. As pacemakers for the circuit, the \ac{AB} and \acp{PD} include $Ca^{2+}$ and $KCa$ channels. 

Several existing models were considered. Very early models from the Hartline laboratory \cite{Hartline1979, Warshaw1976} included up to five neurons, however the decision was made to not consider such very old models but to rather focus on later models that used the Hodgkin-Huxley equations with parameters derived from newer experimental results.

The Golowasch \cite{Golowasch1999a} and Soto-Trevino \cite{Soto-Trevino2005} models, although only modelling two neurons each, proved to be the most comprehensive in terms of available parameters. For instance, by creating a two neuron, two compartment model of the \ac{AB} and \ac{PD} neurons Soto-Trevi$\tilde{n}$o could illustrate how the compartments interact to determine the dynamics of the model neurons \cite{Soto-Trevino2005}.  Both these models use Hodgkin-Huxley equations and were thus relatively easily extendible to incorporate multiple copies of the \acp{PD} and \acp{PY} that would allow modification of gap junction parameter values which is required to model the effect of \ac{DA}. The equations and parameter values from the Golowasch and Soto-Trevi$\tilde{n}$o models were thus used as the starting point for the more complete and biologically accurate pyloric \ac{CPG} model presented. 

As is, the model created for this research incorporates two \acp{PD} neurons which means that it is already possible to use the model for research into the effects of various conditions, such as neuromodulation, of the \acp{PD}. The model is also suitable for investigating the impact of different ranges, differences and ratios of conductance parameters with the intentions of determining the impact of these on the pyloric rhythm. For example such investigations may give a better insight into the mechanisms
 charaterising the \ac{STG} behaviour following decentralisation and during the spontaneous re-establishment of the pyloric rhythm.

The effect of \ac{DA} was modelled by adjusting the values of the $K^{+}$ channel parameters and the gap junction strength between the \acp{PY}. Using t-tests which compared the \ac{ISI} of the five \acp{PY} for gap junction strengths of zero to 90 percent (at 10 percent intervals) against the 100 percent strength value of the gap junctions we were able to confirm that model did indeed succeed in showing the expected de-synchronisation of the \acp{PY} under neuromodulatory conditions.

The model produced can easily be extended to include more neurons, junctions and channels with the main challenge being the finding of appropriate parameters. Such parameters depend on the biological measurements which in turn rely on the methods of recording that are available.

As part of this research we also investigated the use of alternative methods to record the  activity of individual neurons, but in a complete neural circuit. Existing methods are extremely limited in this respect. Methods that can simultaneously record from large number of neurons in such a way that the contribution of individual neurons can be isolated has the advantage of showing the contributions that individual neurons make to the whole. Data recorded in such a way would be invaluable to the development of models as we would be able to parametrise neurons  more accurately and thus reflect biologically true activity.

We investigated three alternative methods. These methods were the use of alternative \acp{VSD} in an attempt to improve responsivity, signal to noise ratio and toxicity, which are the main drawbacks of of current \acp{VSD}. We also looked at the use of \acp{MEA} which could potentially allow not only recording of the complete \ac{STG} but also stimulation of neurons. Lastly we investigated the use of the Picospritzer III for injection of substances such as \ac{VSD} and plasmids into neurons of the \ac{STG} using compressed air. Such injection would have the advantage of offering a much faster and safer delivering method, to improve the longevity of \ac{STNS} preparations. In principle, we were able to show that the three methods could be used successfully with further development.

In conclusion this work presents an improvement on current models of the \ac{STG} with implications for medical science. We still lack conclusive evidence of \acp{CPG} in humans. However, Banaie et. al \cite{Banaie2009} was able to use a model to offer a possible explanation for the gait disorder in \ac{HD}. Unfortunately direct research on vertebrate \acp{CPG} has severe limitations due to ethical and practical constraints. Thus work on invertebrate \acp{CPG} can contribute critically to better understanding of vertebrate movement generation. Our work contributes towards better modelling and understanding of neuromodulatory impact on \ac{CPG} functionality with potential implications to the understanding of neurmodulator-related movement disorders (eg. Parkinsons' Disease). 

Neurons of the same type do not necessarily respond exactly in the same way to neuromodulators such as \ac{DA} and each neuron could affect the network differently. Using the \ac{STG} with the pyloric \ac{CPG} as a model system we have been able to show the need for such differentiation to be considered in computational models. Improved quality data is required to build finer and more detailed models. Here we also reported new methods for improved data analysis and proof of concept for alternative methods for more detailed recording of neurons within networks.

\section{Further work}

The research discussed in this thesis leads to new questions that could be investigated with further research in the future:

\begin{itemize}
	\item We showed that we can model the de-synchronisation of the \acp{PY} caused by \ac{DA}. The question that remains is why is such de-synchronisation required. It is known that the pyloric \ac{CPG} produces alternative rhythms \textit{in vivo}. Hypothetically this de-synchronisation is required to switch to an alternative rhythm.
	\item It is known that there are "early" and "late" \acp{PY}, ie. some \acp{PY} fire before the others. The questions is whether the early and late firing of these neurons is a reflection of the parameters. If that is the case we can hypothesize that the early and late \acp{PY} have different effects on the rhythm and this can potentially be tested with the presented model.
	\item It is known that the pyloric rhythm will stop after de-centralisation, i.e. being disconnected from higher higher ganglia. Recovery of the rhythm has been modelled using Hodgkin-Huxley models by altering the Calcium ion current mechanism. However, it is possible that modification of the maximal conductance values of certain ionic currents might also restore  the rhythmic activity of the pyloric network. It should be possible to test this hypothesis by using the presented model with minor alterations.
\end{itemize}
