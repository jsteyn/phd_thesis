\acresetall
\chapter{Abstract}
\label{chap:Abstract}

\Acp{CPG} are neural circuits that control rhythmic motor patterns such as walking running and swallowing.  Injuries can sever the spinal cord or conditions such as Huntington's disease and Parkinson's disease can damage nerves from the brain that control \acp{CPG}. Understanding the connectivity of neural circuits has proved insufficient to understand the dynamics of such circuits. Neuromodulators and neurohormones can differentially affect every connection in neural circuits and different circuits are affected in very different ways. 


The resulting complexity of such systems make them very difficult to study but research is greatly facilitated by the use of model organisms and computational models. The crustacean \ac{STG} has been used as a model system for many years. Its relative simplicity and accessibility to neurons makes it an ideal system for the study of neural interaction, \acp{CPG} and the effect of neuromodulators on neural systems.

The effect of dopamine on the pyloric \ac{CPG} of the crab \ac{STG} was recorded using voltage sensitive dye imaging and electrophysiological techniques. To analyse \ac{VSD} imaging data a heuristic method was devised that uses the timing of the activity plateaus of neurons for the estimation of the dynamics of the temporal relationship of the neurons' activities.

\matlab was used to create a Hodgkin-Huxley based model of the pyloric constrictor \acp{PD} with parameters that could capture the dynamics of neuromodulation. The \matlab model includes two compartments, the soma and the axon, for the anterior burster neuron, the \acp{LP}, two \acp{PD} and five individual \acp{PY}. 

By differentially changing the values of the model synapses, the model is able to reproduce the de-synchronisation of the pyloric constrictor neurons as was observed experimentally on the deafferented stomatogastric nervous system. Existing models model \acp{PY} and \acp{PD} as single neurons. These models are unable to show the desynchronising effect of dopamine on multiple neurons of the same type. The model created for this research is able to reflect the effect of neuromodulation on the complete circuit by allowing parameters of synapses between neurons of the same type to be adjusted differentially, reflecting the biological system more accurately.
