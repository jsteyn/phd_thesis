\chapter{Simple MATLAB Implementation of the Hodgkin Huxley Equations}
\label{app:HH}

\lstset{ %
	backgroundcolor=\color{white},   % choose the background color; you must add \usepackage{color} or \usepackage{xcolor}
	breakatwhitespace=false,         % sets if automatic breaks should only happen at whitespace
	breaklines=true,                 % sets automatic line breaking
	commentstyle=\color{mygreen},    % comment style
	deletekeywords={...},            % if you want to delete keywords from the given language
	escapeinside={\%*}{*)},          % if you want to add LaTeX within your code
	extendedchars=true,              % lets you use non-ASCII characters; for 8-bits encodings only, does not work with UTF-8
	keepspaces=true,                 % keeps spaces in text, useful for keeping indentation of code (possibly needs columns=flexible)
	keywordstyle=\color{blue},       % keyword style
	language=Matlab,                 % the language of the code
	otherkeywords={*,...},            % if you want to add more keywords to the set
	rulecolor=\color{black},         % if not set, the frame-color may be changed on line-breaks within not-black text (e.g. comments (green here))
	showspaces=false,                % show spaces everywhere adding particular underscores; it overrides 'showstringspaces'
	showstringspaces=false,          % underline spaces within strings only
	showtabs=false,                  % show tabs within strings adding particular underscores
	stringstyle=\color{mymauve},     % string literal style
	tabsize=2,                       % sets default tabsize to 2 spaces
}
\begin{lstlisting}

% Parameters in:
%   v0 		initial value of the potential
%   I_ext 	A 1xT vector. It represent the external 
%			current.
% Constants:
%   DT      ms time steps for ode solver
%   C       Membrane capacitance
%   E_L     mV Leakage reversal potential
%   E_Na    mV Na reversal potential
%   E_K     mV K reversal potentsial
%   G_L     mS/cm^2 Leakage conducatance
%   G_Na    mS/cm^2 Na conductance
%   G_L     mS/cm^2 Leakage conductance
function [v h m n]=HHModel(v0,I_ext)
	T=length(I_ext);
	v=zeros(1,T);
	%Constant values
	DT=0.01; 
	C=0.01; 
	E_L=-49.42; 
	E_Na=55.17; 
	E_K=72.14; 
	G_L=0.003; 
	G_Na=1.20; 
	G_K=0.36; 
	g_L=G_L;
	g_Na=zeros(1,T);
	g_K=zeros(1,T);
	m=zeros(1,T);
	n=zeros(1,T);
	h=zeros(1,T);
	m(1)=0.05;
	h(1)=0.54;
	n(1)=0.34;
	v(1)=v0;
	for t=2:T
		v(t)= v(t-1) + (DT/C)*(I_ext(t-1)-g_L*(v(t-1)-E_L)-g_Na(t-1)* ...
		(v(t-1) -E_Na)-g_K(t-1)*(v(t-1)+E_K));
		m(t)=m_fun(m(t-1),v(t-1),DT);
		n(t)=n_fun(n(t-1),v(t-1),DT);
		h(t)=h_fun(h(t-1),v(t-1),DT);
		g_Na(t)=G_Na*(m(t)^3)*h(t);
		g_K(t)=G_K*(n(t)^4);
	end

function y=alpha_n(V)
	y=(0.1 -0.01*(V+65)) ./(exp(1-0.1*(V+65))-1);

function y=alpha_m(V)
	x=2.5 -0.1*(V+65);
	y=x ./(exp(x)-1);

function y=alpha_h(V)
	y=0.07*exp(-(V+65)/20);

function y=beta_n(V)
	y=0.125 * exp(-(V+65)/80);

function y=beta_m(V)
	y=4 * exp(-(V+65)/18);

function y=beta_h(V)
	y=1 ./(exp(3-0.1*(V+65))+1);

function n=n_fun(n0,v,t)
% n0 is the initial value of n
% v is the constant value of the potential
% t is the time.
	n_0=1 ./(1 + beta_n(v) ./ alpha_n(v));
	tau_n=1 ./(alpha_n(v) +beta_n(v));
	n=n_0 -(n_0 -n0)*exp(-t/tau_n);

function m=m_fun(m0,v,t)
% m0 is the initial value of m
% v is the constant value of the potential
% t is the time.
	m_0=1 ./(1 + beta_m(v) ./ alpha_m(v));
	tau_m=1 ./(alpha_m(v) +beta_m(v));
	m=m_0 -(m_0 -m0)*exp(-t/tau_m);

function h=h_fun(h0,v,t)
% h0 is the initial value of h
% v is the constant value of the potential
% t is the time.
	h_0=1 ./(1 + beta_h(v) ./ alpha_h(v));
	tau_h=1 ./(alpha_h(v) +beta_h(v));
	h=h_0 -(h_0 -h0)*exp(-t/tau_h);


tic
% initialise pulse vector with zeros
I_ext_Pulse=zeros(1,5000);
% set pulse at 10 for time steps 500 to 700 to initiate action potential
I_ext_Pulse(500:700)=10;
v=HHmodel(-65,I_ext_Pulse);
plot(v)
toc
\end{lstlisting}
